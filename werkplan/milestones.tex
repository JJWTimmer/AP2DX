This section will describe the milestones, and what deliverables we will provide. 

\subsection*{Milestone 0}
This will describe what we will have done by the 10th of June. These are the
deliverables:
\begin{description}
\vspace{0cm}
\item[Working environment:] We will set up a Git
repository\footnote{https://github.com/Y3PP3R/AP2DX} and a testing environment
(using jUnit).
\item[Base class:] We will make an abstract base class, on which we can base
all our Java classes. This will contain all the standard methods, e.g. TCP/IP
protocols.
\end{description}

\subsection*{Milestone 1}
This will describe which deliverables we will have done by the 
\begin{description}
\item[Drive:] We want the program to be able to direct
the robot through the environment. We will not yet focus on the ability to
follow lines or walls.
\item[Avoid collision:] The robot should be able to avoid collision with objects
and walls. It will stop, turn a random corner, and drive on. This way it will
cover most of the area without colliding.
\end{description}

Classes needed to be implemented for these goals:
\begin{itemize}
\item Co\"ordinator
\item Sensor
\item Reflexes
\item Motor
\end{itemize}

\subsection*{Milestone 2}
A list of deliverables:
\begin{description}
\item[Avoid obstacles:] The robot will be able to avoid the obstacles that
cross his path, in stead of stopping and turning a random corner.
\item[Navigate:] The robot will be able to navigate through the room.
\end{description}

What we will implement for this:
\begin{description}
\item[Mapper:] A class that creates a map of the room out of the sensor
data. In the time of milestone 2 it does not have to be able to create an entire
map and be very accurate, but it will be able to make some implementation.
\item[Improved Sensor:] The sensor class will be improved to be able to make
an accurate map.
\item[Improved Reflexes:] The reflex class needs to be able to use some
sensor data to be able to avoid objects appropriately. 
\end{description}


\subsection*{Demonstration}
Before the demonstration we will be able to do the following things:
\begin{description}
\item[Planner:] We will have a planner class that can specify directions
based on the current map position and what part of the map we have not yet
discovered.
\item[Improved Mapper:] The mapper will now be able to make an accurate map
and find our location on it, while taking the errors in sensor data into account.
\item[Tests:] Although we test all the time, we want to have tested
everything good before the end.
\item[Report:] We will work on a report, describing our progress, problems
and (test)results.
\item[Documentation:] We will work on a proper documentation of our code,
which is also finished before the end.
\end{description}

The classes we will need to implement or improve for this are:
\begin{itemize}
\item Mapper
\item Planner
\item All test classes
\end{itemize}
