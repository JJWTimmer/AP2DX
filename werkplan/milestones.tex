This section will describe the milestones, and what deliverables we will provide. 

\subsubsection*{Milestone \#0}
This will describe what we have done by the end of week one.

A list of deliverables:
\begin{itemize}
\boldItem{Working environment:} We will have set up a Git
repository\footnote{https://github.com/Y3PP3R/AP2DX} and a testing environment
(using xUnit).
\boldItem{Base class:} We will make an abstract base class, on which we can base
all our Java classes. This will have all the standard things, like TCP/IP
protocols.
\end{itemize}

\subsubsection*{Milestone \#1}
A list of deliverables:
\begin{itemize}
\boldItem{Drive around:} We want the program to be able to direct
the robot through the environment. We will not yet focus on the ability to
follow lines or walls.
\boldItem{Avoid collision:} The robot should be able to avoid collision with objects
and walls. It will stop, turn a random corner, and drive on. This way it will
cover most of the area without colliding.
\end{itemize}

Classes needed to be implemented for this:
\begin{itemize}
\item Co\"ordinator
\item Sensor
\item Reflexes
\item Motor
\end{itemize}

\subsubsection*{Milestone \#2}
A list of deliverables:
\begin{itemize}
\boldItem{Avoid obstacles:} The robot will be able to avoid the obstacles that
cross his path, in stead of stopping and turning a random corner.
\boldItem{Navigate:} The robot will be able to navigate through the room.
\end{itemize}

What we will implement for this:
\begin{itemize}
\boldItem{Mapper:} A class that creates a map of the room out of the sensor
data. In the time of milestone 2 it does not have to be able to create an entire
map and be very accurate, but it will be able to make some implementation.
\boldItem{Improved Sensor:} The sensor class will be improved to be able to make
an accurate map.
\boldItem{Improved Reflexes:} The reflex class needs to be able to use some
sensor data to be able to avoid objects appropriately. 
\end{itemize}


\subsubsection*{Demonstration}
Before the demonstration we will be able to do the following things:
\begin{itemize}
\boldItem{Planner:} We will have a planner class that can specify directions
based on the current map position and what part of the map we have not yet
discovered.
\boldItem{Improved Mapper:} The mapper will now be able to make an accurate map
and find our location on it, while taking the errors in sensor data into account.
\boldItem{Tests:} Although we test all the time, we want to have tested
everything good before the end.
\boldItem{Report:} We will work on a report, describing our progress, problems
and (test)results.
\boldItem{Documentation:} We will work on a proper documentation of our code,
which is also finished before the end.
\end{itemize}

The classes we will need to implement or improve for this are:
\begin{itemize}
\item Mapper
\item Planner
\item All test classes
\end{itemize}
