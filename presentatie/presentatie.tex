\documentclass{beamer}
\usepackage[all]{xy}
\usepackage[utf8x]{inputenc}
\usepackage{default}
\usepackage{verbatim}

\title{AP2DX}
\subtitle{Awesomizing the P2DX}
\author{Jasper Timmer, Maarten Inja, Maarten de Waard, Wadie Assal}
\institute{UvA}
\usetheme{Warsaw}

\begin{document}
\begin{frame}
\titlepage
\end{frame}

\begin{frame}
\tableofcontents
\end{frame}


\section{Introduction}
\begin{frame}
\frametitle{Introduction}
\begin{itemize}
\item The same assignment
\item Who we are
%Hier dacht ik aan alvast vertellen over Unit tests
\item What is special in our case
\end{itemize}
\end{frame}

\begin{frame}
\frametitle{Goals}
\begin{itemize}
\item Loosely coupled modules based on network communication
\item Robot should be safe, i.e. stop for obstacles
\item Robot should be able to drive autonomously through the environment
\item Robot should be able to create a map of the environment
\item No user input will be required
\end{itemize}
\end{frame}

\section{Architecture}
\subsection{Framework}
\begin{frame}
\frametitle{Architecture - the basics}
\xymatrix{& USARSim \ar@/_/[d] &  \\
& Coordinator \ar@/_/[u] \ar[dr] & \\
Abstract Motor \ar[ur] & Reflex \ar[l]  & Sensor \ar[l] \ar[d] \\
& Planner \ar[u] & Mapper \ar[l] }
\end{frame}


\begin{frame}
\frametitle{Architecture - into the depths}
%\\\\\\
\includegraphics[width=.8\paperwidth]{AP2DX}
\end{frame}

\begin{frame}
\frametitle{Abstract base class}
We decided to use an abstract class, to base all our classes on.
\begin{block}{Advantages}
\begin{itemize}
\item Very easy to work with
\item Only have to make the connection protocol once
\item Strict contracts with team mates
\end{itemize}
\end{block}
\begin{block}{Disadvantages}
\begin{itemize}
\item Stuck with one language
\item Hard debugging
\item Hard to make big changes, or add things we had not thought of
\end{itemize}
\end{block}
\end{frame}

\subsection{Movement and collision detection}
\begin{frame}
\frametitle{Movement - how we do it}
\begin{itemize}
    \item Based on Sonar sensor data
    \item Not using the outer two sonars (-90$^{\circ}$ and 90$^{\circ}$)
    \begin{itemize}
        \item Only use them for detecting ``gaps in the wall'', places we may
        want to go
        \item ``hey, this time the wall is $>$ 1 meter further away in one tick, maybe there is a hole!''
        \item  Turn to the hole, scan the width, and decide to drive through or not.
    \end{itemize}
\end{itemize}
\end{frame}

\begin{frame}
    \frametitle{Movement - reflexes}
    \begin{itemize}
        \item If reflex stops the robot, because of an object in the way, turn
        to the direction with the furthest view angle
        \item If all four middle sonar sensors detect $>$ 0.5m, we can drive
        forward.
        \begin{itemize}
            \item The bot's width is 38cm, angle between outer-middle 2 sonars (sensor 3 and 6) = 60 degrees, lets say the hole needs to be 45cm wide to drive through it, those two sensors need both to see at least 45cm. (cosine rule)
        \end{itemize}
    \end{itemize}
\end{frame}



\subsection{Creating a map}
\begin{frame}
\frametitle{Mapper}
We did not make our own mapper. We used DP
Slam\footnote{http://www.cs.duke.edu/~parr/dpslam/}\footnote{Algorithm from: Austin Eliazar,
Ronald Parr: DP-SLAM: Fast, Robust Simultainous Localization and Mapping Without
Predetermined Landmarks}.\\
The program works in C, and creates a map like this one:
\includegraphics[height=.5\textheight]{FirstHighMap}
\end{frame}

\begin{frame}
\frametitle{Mapper - What makes it special}
\begin{itemize}
    \item Two ways to use a mapper:
    \begin{itemize}
        \item While driving
        \item After driving (with saved sensordata)
    \end{itemize}
    \item We make a map, while driving
    \item Mapper uses Odometry and Laser range scanner data
    \item Currently only works on linux
    \end{itemize}
\end{frame}

\subsection{Communication}
\begin{frame}
\frametitle{Messages}
\xymatrix{& Message \ar[dl] \ar[dr] &  \\
AP2DXMessage \ar[d] && UsarSimMessage \ar[d] & \\
SpecializedMessage\ar[d] && (individual messages) \\
(individual messages)& &  }
\end{frame}

\begin{frame}
\frametitle{Messages - explained}
There are a couple of advantages and disadvantages:
\begin{block}{Advantages}
\begin{itemize}
\item Very easy to work with
\item Easy to add a new kind of message
\item Strict restrictions to how a message should look (and thus uniformity)
\end{itemize}
\end{block}

\begin{block}{Disadvantages}
\begin{itemize}
\item Very hard to debug
\item Hard to add a type of message that doesn't fit in
\end{itemize}
\end{block}
\end{frame}

\section{Developing process}
\begin{frame}
\frametitle{Developing process - 1}
\begin{block}{Test driven programming}
\begin{itemize}
    \item Unit test in front
    \begin{itemize}
        \item Smallest testable part of code
        \item Every dependency is mocked
    \end{itemize}
    \item A lot of overhead for small project
\end{itemize}
\end{block}
\begin{block}{File management}
We used Git, to easily synchronize our code. There are some advantages:
\begin{itemize}
    \item Distributed, so it is not like subversion
    \item It has change tracking
    \item It automatically merges
    \item It is installed on the UvA computers.
\end{itemize}
\end{block}
\end{frame}

\begin{frame}
\frametitle{Developing process - 2}
\begin{block}{Automatic building}
    Ant was used to build everything. This includes compiling, testing,
    publishing test reports, javaDoc, creating jar files.
\end{block}
\begin{block}{Continuous integration server}
Jenkins was used as integration server. It uses:
\begin{itemize}
    \item Git checkout
    \item Ant build
    \item Clarifying coverage reports
\end{itemize}
Together this makes a nice way to discover every detail about our project.
Screenshots will follow.
\end{block}
\end{frame}

{
     \usebackgroundtemplate{\includegraphics[width=\paperwidth]{jenkins_project}} 
      \begin{frame}
      \end{frame}
}

{
     \usebackgroundtemplate{\includegraphics[width=\paperwidth]{jenkins_coverage}} 
      \begin{frame}
      \end{frame}
}
{
     \usebackgroundtemplate{\includegraphics[width=\paperwidth]{jenkins_linecoverage}} 
      \begin{frame}
      \end{frame}
}

\section{Discussion}
\begin{frame}
\frametitle{Discussion}
\begin{itemize}
\item Framework is handy and dynamic, due to the JSon config files.
Nevertheless, if you would add an extra module, some code would have to change.
\item Framework took a lot of time to set up and debug
\item Messages took a lot of time to debug, because we were using JSon
\end{itemize}
\end{frame}

\end{document}
