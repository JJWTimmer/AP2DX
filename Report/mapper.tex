The task of the mapper module is to create a map of the environment and to find the position of
the robot in this map. This will allow the planner module to plan paths and other behaviour, that is:
if it is done correctly! 

Luckily, simultaneously localizing and mapping (SLAM) 
%creating a map and simultaneously localizing a robot 
is a commonly encountered problem and 
many research has been conducted to optimize algorithms. The mapper module uses the \emph{DP-SLAM}
\footnote{\url{http://www.cs.duke.edu/~parr/dpslam/}} implementation in the C programming language.
Starting the DP-SLAM program (which was slightly modified for compatibility with the Java module)
is done with a system call, after which two threads are started. One to write the correct sensor 
data to the input stream of the program, and one to read the output stream of the program. The 
output stream contains information about for example the location of the robot, whether or not
something went wrong or when a new map has been dumped to file.

Unfortunately, the mapper module is not multi platform as the DP-SLAM program uses Linux libraries. 
If this module is run in a non Windows system no map can be created, but sensor important data will
still be forwarded to the planner module. 


