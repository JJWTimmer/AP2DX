% Misschien moeten hier wat subsecties bij, kan bovendien wel netter taal gebruik worden
\subsection{Protocols}
There are two seperate protocols in use in AP2DX. The first protocol is the Gamebot protocol 
used by Unreal Tournament to communicate with the simulation. The coordinator module is the 
only module that uses these messages. First the coordinator module sends a ``spawn'' message 
to create the robot in the simulator. Then it continues to read messages (often containing 
sensor data) send by the simulator and to send messages (containing ``drive'' commands) to
the simulator. 
\\\\
The second protocol is the AP2DX message protocol. These are JSON strings that \emph{always}
contain a source module ID, a destination module ID and lastly a type, indicating what kind
of message this is. Other data in these messages might vary. For example; a sonar sensor
message has the sonar data in it. A drive message only stores the information on how to drive
in the string. 
This protocol is used to communicate between all the modules in 
AP2DX.  The coordinator module is responsible for translating messages using a kind of
protocol to a message using the other kind of protocol.
\\\\
Using two protocols meant that a lot of things had to be done twice, but also that a module 
could deal with both protocols. Therefor the 
base class deals with instances
of our abstract ``Message'' class and has an interface ``IMessageParser''. Final modules \emph{know}
what kind of protocol a connection uses and can set up the right message parser (``AP2DXMessageParser'' 
or ``USARSimMessageParser'') and thus, deal correctly with the incoming messages. 
\\\\
Different kind of messages need to be dealt with as well. The source and destination module ID take care
of getting the message to the right module. The type is used to identify a message. Once a message
has been identified as, for example a range scanner sensor message (type: AP2DX\_SENSOR\_RANGESCANNER), 
it becomes a ``specialized message''. 
A specialized message parses the variables in the string and sets them in variables in the message. Now,
simple setters and getters can be used to modify and read all the different 
\\\\
The hierarchy of our messages can be seen in figure X. 
\todo{create and insert picture here}
(picture; Message to AP2DXMessage and USARSimMessage. AP2DXMessage to SpecializedMessage. SpecializedMessage
to all the sensor messages and motor messages and drive messages. USARSimMessage to Jaspers messages) 

% add this link somewhere, for kicks and giggles: http://udn.epicgames.com/Two/UnrealScriptReference.html#Variables
\subsection{Synchronization between Reflex and Planner}
All modules internally use a queue to store all the received messages. Those messages are processed one by one. As soon as the reflex sends a stop message to the motor based on the distance threshold, it starts ignoring all other motor messages it gets from the planner. The planner has to work through it's queue until it reaches the stop message. At this moment, it sends a reset message to the reflex, signaling: everything from now on is meant to get us away from the wall. The Reflex sets a 'getting away from wall' flag and starts relaying again, but is does not watch the distance threshold. When the Planner finished the turn away procedure, it sends a clear message to the Reflex, and from that moment, the robot is again cruising. The Reflex will reset internal state and start again with watching the threshold.