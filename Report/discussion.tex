\subsection{Architecture}
The architecture of AP2DX was well designed. What we thought off in the first week
was nearly exactly programmed when everything started to work (give or take a few
small details, which were easy to add). The framework is robust and more functionallity
can easily be added. For example; sensor messages we do not use can be added in mere moments 
by creating a specizialed message. 

But, the question is: Is this a good thing? Obviously it is nice to see the framework working
and to know new functions can be added later. But it took us a loot of time to set it up. We 
did not need to parse all kind of special messages if only use two or three! 

Also the base class disallowed us to test anything with the simulator until it was completely 
finished. Each module could easily be programmed in another programming language, but this
pinned us down to use Java, and the base class. 

\subsection{Message Protocols}
Parsing messages \\
We originally thought the messages for USAR and AP2DX were quite similar, but they diverged more
and more the further we came to working code. In the end it simply seemed as if we did the same
thing twice, with no real advantage. The USAR messages had to be parsed either way. They also
had to be compiled either way; there was no alternative method to communicate with the server. 
In hindsight, we might have better programmed that well, with a nice interface to set and get
variables rather than to program it twice. 

\subsection{Mapper}
We downloaded a complicated, sophisticated mapper from the internet. This is
good, but the mapper does assume that you always give it correct information.
This way, the mapper does not always work.

Also our sophisticated mapper has a slight delay, therefore it misses some
information we would put in it.

In hindsight we might have gotten a better mapper if we had downloaded a simpler
implementation, and had used that with our program. That would have been better
at mapping while driving.

\subsection{Autonomous system}
As AI-people, we have built an entirely autonomous system. This means that our
robot can go where it wants, ``when it wants''. This is nice, but it does not
come in handy, when you want it to do what you want. Sometimes it would have
been nice to be able to provide some user input. 

\subsection{Slow}
Our program turned out to be a bit slow, over network. This could be because we
print a lot of data, and because every communication is over network. Still,
while running everything on one computer, the robot did not crash. This means
that the program is fast enough.

