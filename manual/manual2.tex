\documentclass[a4paper, notitlepage]{article}

\title{AP2DX Manual}
\author{Wadie Assal \\ 6398693 \and Jasper Timmer \\ 5995140 \and Maarten de Waard \\ 5894883 \and Maarten Inja (vz) \\ 5872464}

\begin{document}

\maketitle

\section{Intro}
This manual describes how to run and configure the complete AP2DX package. The simulated P2DX robot now maps its environment and has three behaviors: 
\begin{itemize}
\item Curiously driving through holes in the sensor data. If it detects a hallway, why should not the robot enter and explore it? 
\item Cautiously drive away from walls it might bump into. Of course, we would not want the robot to mess up the odometry sensor data by slipping it wheels, 
trying to drive through a wall it can't. Instead it stop, turns and wanders some other way.
\item Sometimes even the best of us can't avoid ending up stuck inside a corpse. Reverse direction and get out of there! Afterwards, 
when P2DX is once again in the clear.
\end{itemize}

\section{Requirements}
To Run AP2DX Software you need Java Runtime Enviroment 1.6. If one wants to map the robots environment on the go 
the mapper module should be run on a Linux system. If the mapper is not run on a Linux system no map will be created
and the mapper modules window will remain creepy white. 

\section{How to configure}
A module consists of a folder with in this folder four files: 
a .jar file, a .json file and two ``start'' scripts to run it on Windows and Linux. The .json file is the configuration 
file where one should change IP addresses of other modules (or the simulator) and ports. 

\section{How to built}
The build.xml contains an ant script that compiles to source code. 
One optional argument can be given: ``debug'' which only compiles and 
creates .jar files, without the javadoc compilation and the jUnit testing phases.

\section{How to run}
To start the AP2DX Software, you first need to run an USARsim Server. Before running the AP2DX software be sure the configuration
files correctly list the IP addresses and the ports of the other modules and, very importantly, the USARsim Server. 

Next, the modules have to be run. If one runs all the modules on one computer one can run start\_all.bat or if one runs Linux: start\_all.sh. 
If one runs the modules separately run the start.bat or start.sh files next to the jar files in the following order.
\begin{itemize}
\item Coordinator
\item Motor
\item Reflex
\item Planner
\item Mapper
\item Sensor
\end{itemize}


\end{document}
